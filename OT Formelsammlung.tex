\input{header.tex}


\begin{document}

\maketitle

Dieser Text ist unter dieser \href{http://creativecommons.org/licenses/by-nc-sa/4.0/}{Creative Commons} Lizenz veröffentlicht.

\textcolor{red}{Ich erhebe keinen Anspruch auf Vollständigkeit oder Richtigkeit. Falls ihr Fehler findet oder etwas fehlt, dann meldet euch bitte über den Emailkontakt.}

\tableofcontents

\newpage


\section{Linsen}

\subsection*{Allgemein}

\begin{align*}
\frac{1}{f} &= \frac{1}{b} + \frac{1}{g}  \\
m &= - \frac{b}{g}
\intertext{Linsen (dünn) mit Abstand:}
\frac{1}{f} &= \frac{1}{b} + \frac{1}{g} - \frac{d}{f_1 f_2}
\intertext{Linsenschleiferfromel für dünne Linsen:}
\frac{1}{f} &= \left( n - 1 \right) \cdot \left( \frac{1}{r_1} - \frac{1}{r_2} \right)
\intertext{Linsenschleiferfromel für dicke Linsen:}
\frac{1}{f} &= \left( n - 1 \right) \cdot \left( \frac{1}{r_1} - \frac{1}{r_2}  + \frac{(n - 1) \cdot d}{r_1 \cdot r_2 \cdot n}\right)
\intertext{Brennpunkt:}
s'' &= \frac{f_2 \cdot \left( f_1 - d \right)}{f_1 + f_2 - d}
\end{align*}

\subsection*{Lupe}

\begin{align*}
V_L &= \frac{\epsilon}{\epsilon_0} = \frac{S_0}{f} + 1
\end{align*}

Dabei ist $\epsilon_0$ der Sehwinkel ohne Lupe, $\epsilon$ der mit Lupe und $S_0$ der Abstand vom Nahpunkt zum Auge.


\subsection*{Microskop}

\begin{align*}
\intertext{Abbildungsmaßstab}
V_{Ob} &= \frac{B}{G} = \frac{-t}{f_{Ob}}
\intertext{Winkelvergrößerung}
\nu_{Ok} &= \frac{S_0}{f_{Ok}}
\intertext{Gesamtvergrößerung}
\nu_M &= V_{Ob} \cdot \nu_{Ok} = - \frac{t}{f_{Ob}} \cdot \frac{S_0}{f_{Ok}}
\end{align*}



\subsection*{Teleskop}


\begin{figure}[h]
	\centering
	\includegraphics[scale=0.9]{Teleskop.jpg}
\end{figure}

\begin{align*}
\tan \left( \epsilon_{Ob} \right) &= - \frac{B}{f_{Ob}} \approx \epsilon_{Ob} \\
\tan \left( \epsilon_{Ok} \right) &= - \frac{B}{f_{Ok}} \approx \epsilon_{Ok}
\intertext{Vergrößerung}
\nu_T &= \frac{\epsilon_{Ok}}{\epsilon_{Ob}} = - \frac{f_{Ob}}{f_{Ok}}
\end{align*}


\subsection*{Kamera}

\begin{align*}
Blendenzahl = \frac{f}{d}
\end{align*}



\section{Polarsation}

\begin{align*}
\intertext{Allgemeine Transmission:}
T_\perp &= e^{-\mu_\perp \cdot d} \\
T_\parallel &= e^{-\mu_\parallel \cdot d} 
\intertext{Dicke eine Lambdaviertelplatte:}
d &= \frac{\lambda}{4 \cdot |n_o - n_e|} \\
o: E_1(t) &= E_0 \sin(\phi) \cos(\omega t) \\
ao: o: E_2(t) &= E_0 \cos(\phi) \sin(\omega t) \\
I &= \frac{c}{2n} \epsilon_0 E_0^2
\end{align*}

\newpage

\section{Reflexionen / Transmission}

\begin{align*}
\intertext{Die Dicke einer Antireflexschicht ist:}
d &= \frac{\lambda_0}{4 \cdot n_{AR}}
\intertext{Das erzeugt einen Gangunterschied von:}
\phi &= \frac{2 \pi}{\lambda} \cdot \frac{\lambda'}{2}
\intertext{Dabei ist $\lambda'$ die Wellenlänge gegen die die Antireflexschicht wirkt und $\lambda$ die eingestrahlte Wellenlänge.}
\intertext{Extinkt./ optische Dichte}
E_\lambda &= -\lg\left( \frac{I}{I_0} \right) = \epsilon_\lambda \cdot c \cdot d
\intertext{Lambert-Beersches Gesetz:}
I(d) &= I_0 \cdot e^{-\epsilon c d}
\end{align*}

Für die Transmission gilt:

\begin{align*}
\text{grob} \ T &\approx \left( 1- R \right)^2 \cdot \tau \qquad \text{exact} \ T = \frac{\left( 1 - R \right)^2 \cdot \tau}{1 - \left( R \tau \right)^2} \\ 
\text{mit} &\qquad \tau = e^{-Kd} 
\intertext{Für die Oberflächenreflexion gilt:}
R &= \left( \frac{n - 1}{n + 1} \right)^2
\end{align*}

Für den Brechungsindex einer Antireflexschicht auf einem Medium gilt:

\begin{align*}
n_{AR} &= \sqrt{n_{Medium}}
\end{align*}

Totalreflexion:

\begin{align*}
\theta_c = \arcsin\left( \frac{n_2}{n_1}\right)
\end{align*}


\newpage


\section{Lichtleiter}

\begin{align*}
\intertext{Numersche Apertur}
N_A &= \sqrt{n_{core}^2 - n_{clad}^2} = n_{aus} \cdot \sin(\theta_{aus})
\end{align*}


\section{Geometrische Optik}


\begin{align*}
\intertext{Sphärischer Spiegel}
\frac{1}{g} &+ \frac{1}{b} = \frac{2}{r} = \frac{1}{f}
\intertext{Abbildung an spärischen Flächen}
\frac{n_1}{g} + \frac{n_2}{b} &= \frac{n_2 - n_1}{r} \qquad n_2 > n_1 \\
m &= \frac{B}{G} = - \frac{n_1 \cdot b}{n_2 \cdot g}
\intertext{Besselverfahren:}
d &= g + b - b_{anst} \\
f &= \frac{1}{4} \cdot \frac{d^2 - \Delta^2}{d}
\intertext{Bildfeldwölbung:}
\Delta x &= \frac{y^2}{2} \cdot \frac{1}{nf} 
\intertext{mit y als Abstand Punkt - optische Achse und $\Delta x$ als Abstand Punkt - Parabel Schirm}
\intertext{Abbezahle:}
\nu &= \frac{n_{e} - 1}{n_{F'} - n_{C'}} \\
\end{align*}


\section{Beugung}

\begin{align*}
\intertext{Intensität am Spalt:}
I(\theta) &= \frac{\sin^2\left( \frac{\pi b \sin(\theta)}{\lambda} \right)}{\left( \frac{\pi b \sin(\theta)}{\lambda} \right)^2}
\intertext{Im ersten Minimum gilt:}
\pi &= \frac{\pi b \sin(\theta)}{\lambda}
\intertext{Für kleine Winkel gilt:}
\tan(\theta) &= \theta = \sin(\theta) = \frac{\lambda}{b}
\intertext{Wir definieren $\Delta x$ als den Abstand vom zentralen Maximum zum ersten Minumum:}
\Delta x &= L \cdot \frac{\lambda}{b}
\end{align*}

Für eine Lochblende gilt:

\begin{align*}
I(\theta) \sim \left[ \frac{2 j_1 \cdot \left( \pi d \cdot \frac{\sin(\theta)}{\lambda} \right)}{\pi d \frac{\sin(\theta)}{\lambda}} \right]^2
\intertext{Der erste dunkle Ring entspricht nun der ersten Nullstelle von $j_1$. Aus der Vorlesund wissen wir das dies bei $x = 1,22 \pi$ der Fall ist:}
1,22 \pi &= \pi d \cdot \frac{\sin(\theta)}{\lambda} \\
\Leftrightarrow \sin(\theta) &= 1,22 \cdot \frac{\lambda}{d} \approx \theta
\end{align*}

Dabei ist $d$ der Durchmesser des Lochs.





\section{Licht}


\begin{align*}
\intertext{Lichtstrom:}
\phi &= K(\lambda) \cdot V(\lambda) \cdot \phi_e
\intertext{Strahlstärke:}
I_e &= \frac{\d \phi_e}{\d \Omega} = \frac{\phi_e}{\underbrace{4 \pi}_{alle \ Raumrichtungen}}
\intertext{Lichtstärke}
I &= \frac{\phi}{\underbrace{4 \pi}_{alle \ Raumrichtungen}}
\intertext{spezifische Ausstrahlung}
M_e &= \frac{\phi_e}{A}
\intertext{Bestrahlungsstärke:}
E_e &= \frac{\d \phi_e}{\d A}
\intertext{Beleuchtungsstärke:}
E &= K_m \cdot V(\lambda) \cdot E_e
\intertext{Strahlungsfluss:}
\phi_e &= E_e \cdot A
\intertext{Lichtstrom:}
\phi &= K \cdot V(\lambda) \cdot \phi_e
\end{align*}



























\end{document}
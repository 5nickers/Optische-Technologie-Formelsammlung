\input{header.tex}


\begin{document}

\maketitle

Dieser Text ist unter dieser \href{http://creativecommons.org/licenses/by-nc-sa/4.0/}{Creative Commons} Lizenz veröffentlicht.

\textcolor{red}{Ich erhebe keinen Anspruch auf Vollständigkeit oder Richtigkeit. Falls ihr Fehler findet oder etwas fehlt, dann meldet euch bitte über den Emailkontakt.}

\tableofcontents


\newpage


\section{Linsen}

\begin{align*}
\frac{1}{f} &= \frac{1}{b} + \frac{1}{g}  \\
m &= - \frac{b}{g}
\intertext{Linsenschleiferfromel für dünne Linsen:}
\frac{1}{f} &= \left( n - 1 \right) \cdot \left( \frac{1}{r_1} - \frac{1}{r_2} \right)
\intertext{Linsenschleiferfromel für dicke Linsen:}
\frac{1}{f} &= \left( n - 1 \right) \cdot \left( \frac{1}{r_1} - \frac{1}{r_2}  + \frac{(n - 1) \cdot d}{r_1 \cdot r_2 \cdot n}\right)
\end{align*}



\section{Polarsation}

\begin{align*}
\intertext{Allgemeine Transmission:}
T_\perp &= e^{-\mu_\perp \cdot d} \\
T_\parallel &= e^{-\mu_\parallel \cdot d} 
\intertext{Dicke eine Lambdaviertelplatte:}
d &= \frac{\lambda}{4 \cdot |n_o - n_e|}
\end{align*}

Falls mit den Stokes Matrizen gerechnet werden soll bekommen wir die Tabelle dazu. Wichtig ist, das die Matrizen in umgekehrter Reihenfolge des Lichtwegs miteinander multipliziert werden.

\section{Reflexionen / Transmission}

\begin{align*}
\intertext{Die Dicke einer Antireflexschicht ist:}
d &= \frac{\lambda_0}{4 \cdot n_{AR}}
\intertext{Das erzeugt einen Gangunterschied von:}
\phi &= \frac{2 \pi}{\lambda} \cdot \frac{\lambda'}{2}
\intertext{Dabei ist $\lambda'$ die Wellenlänge gegen die die Antireflexschicht wirkt und $\lambda$ die eingestrahlte Wellenlänge.}
\end{align*}

Für die innere Transmission gilt:

\begin{align*}
T &= \frac{\left( 1 - R \right)^2 \cdot \tau}{1 - \left( R \tau \right)^2} \qquad \text{mit} \qquad \tau = e^{-Kd} 
\intertext{Für die Oberflächenreflexion gilt:}
R &= \left( \frac{n - 1}{n + 1} \right)^2
\end{align*}

Für den Brechungsindex einer Antireflexschicht auf einem Medium gilt:

\begin{align*}
n_{AR} &= \sqrt{n_{Medium}}
\end{align*}




\section{Beugung}

\begin{align*}
\intertext{Intensität am Spalt:}
I(\theta) &= \frac{\sin^2\left( \frac{\pi b \sin(\theta)}{\lambda} \right)}{\left( \frac{\pi b \sin(\theta)}{\lambda} \right)^2}
\intertext{Im ersten Minimum gilt:}
\pi &= \frac{\pi b \sin(\theta)}{\lambda}
\intertext{Für kleine Winkel gilt:}
\tan(\theta) &= \theta = \sin(\theta) = \frac{\lambda}{b}
\intertext{Wir definieren $\Delta x$ als den Abstand vom zentralen Maximum zum ersten Minumum:}
\Delta x &= L \cdot \frac{\lambda}{b}
\end{align*}

Für eine Lochblende gilt:

\begin{align*}
I(\theta) \sim \left[ \frac{2 j_1 \cdot \left( \pi d \cdot \frac{\sin(\theta)}{\lambda} \right)}{\pi d \frac{\sin(\theta)}{\lambda}} \right]^2
\intertext{Der erste dunkle Ring entspricht nun der ersten Nullstelle von $j_1$. Aus der Vorlesund wissen wir das dies bei $x = 1,22 \pi$ der Fall ist:}
1,22 \pi &= \pi d \cdot \frac{\sin(\theta)}{\lambda} \\
\Leftrightarrow \sin(\theta) &= 1,22 \cdot \frac{\lambda}{d} \approx \theta
\end{align*}

Dabei ist $d$ der Durchmesser des Lochs.













\end{document}